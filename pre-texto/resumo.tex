% Resumo
\begin{resumo}
% Diminuir espaçamento entre título e texto
\vspace{-1cm}

% Texto do resumo: sem paragrafo, justificado, com espaçamento 1,5 cm
\onehalfspacing

\noindent 
  Com o avanço tecnológico cada vez mais presentes no cotidiano das pessoas, a informatização de processos manuais se faz mais necessário em algumas organizações. Portanto, este trabalho tem como objetivo criar um aplicativo de \textit{smartphone}, que realiza empréstimos dos equipamentos do laboratório de fotografia e áudio visual da PUC Minas, facilitando o processo de empréstimo e o tornando mais ágil e eficiente. Para o desenvolvimento do aplicativo, foi utilizado uma tecnologia multiplataformas, permitindo a construção de um único código para diversas plataformas \textit{mobile}. O trabalho apresenta conceitos e características do desenvolvimento \textit{mobile}, tais como vantagens, desvantagens, peculiaridades, dentre outros pontos importantes para alcançar o objetivo geral do trabalho. A construção da aplicação deu se por etapas, partindo de um entendimento geral do escopo e as necessidades dos envolvidos junto a análise de requisitos para definir quais as funcionalidades deveriam ser implementas, definição da arquitetura e prototipagem do aplicativo, em seguida o desenvolvimento, testes e o processo de homologação para validar a implementação das funcionalidades e simular o funcionamento da aplicação em ambiente real.

% Espaçamento para as palavras-chave
\vspace*{.75cm}

% Palavras-chave: sem parágrafo, alinhado à esquerda
\noindent Palavras-chave: Mobile, Desenvolvimento, React Native, Multiplataforma
% Segunda linha de palavras-chave, com espaçamento.
%\indent\hspace{2cm}Palavra.

\end{resumo}